\chapter{Evaluation of Employment}

Having studied at UQ and worked at GBST in parallel for over 3 years, I've gained a useful insight into how the skills and processes we study are applied to the workplace. 
\\

My understanding of agile development practices has been greatly enriched by my experience at GBST, and serves as a good example of how I have benefited by working with them for far longer than the required professional practice placement time. As the team I worked in grew, it no longer fit a simplified Scrum model too well. We had a large number of products, a large number of part-time employees and an uneven distribution of product familiarity/knowledge across the team. Some of the changes we made to adapt the scrum methodology to our specific needs over time included:
\begin{itemize}
\item trialling several formats of sprint retrospectives
\item extending sprint lengths from 2 to 3 weeks, in order to improve efficiency of sprint meetings
\item introduced regular process improvement meetings
\item daily handover of tasks for part-time employees
\item splitting our daily stand-up meetings into 'silos', grouped by product area, to optimise time spent
\end{itemize}

In general, much of the learning I undertook at GBST was synergetic with my coursework at UQ, and vice versa. In the semester immediately before starting employment, I completed the introductory database course at UQ (INFS1200). In my role, I was required to edit, write and optimise SQL queries regularly, operate on large databases, and convert queries between SQL dialects. In my remaining semesters of study, I found all tasks requiring me to write SQL queries trivially easy.

This pattern - being introduced to a technology by UQ, gaining proficiency through practice at GBST, and utilising the skills I learned in my studies again - was regularly repeated across most of the programming languages I worked with (Bash scripts, SQL, Java etc), tools (SVN, Git, Linux servers, VirtualBox, VIM) and IDEs (Eclipse, IntelliJ IDEA).

As the company develops software for the finance sector, I also had the opportunity to gain insights into that industry, including passive learning and active training on financial markets and terminology.
\\

Overall, I was very fortunate to be approached by GBST in my first year of university, as it's given me extensive exposure to an engineering workplace, prolonged practice for some of my fields of study, and a demonstration of how to maintain and adapt good software engineering processes over time.

%======================================================================
\begin{comment}
lorem ipsum blah blah blah
\end{comment}
